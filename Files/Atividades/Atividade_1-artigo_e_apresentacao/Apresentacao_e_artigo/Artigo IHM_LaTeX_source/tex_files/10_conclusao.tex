\chapter{Conclusão}

Tendo em vista o que foi revisado em outras literaturas vê-se necessário que os software desenvolvidos com a finalidade de solucionar o problema de grade horária deve ter enfoque no usuário, visto que dada a complexidade intrínseca à multidimensionalidade do problema, acaba que o software tende a se tornar similarmente complexo, assim se tornando difícil de se manusear, causando então a tendência do usuário de se ater ao método vigente de resolução.