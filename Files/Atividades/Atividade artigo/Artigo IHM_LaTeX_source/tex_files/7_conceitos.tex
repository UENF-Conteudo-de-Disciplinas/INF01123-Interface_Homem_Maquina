\chapter{Conceitualização}

Antes de maiores aprofundamentos é necessário elucidar melhor alguns conceitos. Referentes à esta área de pesquisa.

\section{Agendamento de grade horária e seus termos}

    Nesta tarefa de definições de termos envolvendo grades horárias, Wren \cite{WREN1996} realiza um bom trabalho ao definir alguns termos relevantes ao tópico e suas definições serão utilizadas como norte dos conceitos abaixo listados.
    
    \subsection{\textit{Scheduling} (agendamento)}
    
        Segundo Wren: \cite{WREN1996}
        
        \begin{quote}\footnotesize
            O \textit{scheduling} pode ser visto como a organização de objetos em um padrão no tempo ou no espaço, de forma que algumas metas sejam atingidas, ou quase atingidas, e que as restrições sobre a forma como os objetos podem ser organizados sejam satisfeitas, ou quase satisfeitas.
            
            Os objetos podem ser pessoas, veículos, aulas, exames, máquinas, trabalhos em uma fábrica, etc. Muitas vezes, a formação de objetos pode ser vista como parte do processo de programação.
        \end{quote}
        
        Sendo assim, podemos considerar que o agendamento é um caso geral de organização e alocação de recursos segundo certos limites e objetivos.
    
    \subsection{Restrições (\textit{constraints})}
    
        Também segundo Wren: \cite{WREN1996}
        
        
        \begin{quote}\footnotesize
            As restrições definem relações físicas ou legais entre os objetos ou entre os objetos, outros objetos e o padrão. Elas determinam as maneiras pelas quais os objetos podem ser encaixados uns nos outros ou no padrão. As restrições podem ser vistas como regras que impedem a realização das metas. No entanto, elas também podem ser vistas como parte da especificação do problema, que pode ser usada para orientar o solucionador em direção a uma solução. Algumas restrições podem existir apenas na visão do proprietário do problema, e pode fazer parte do processo de solução indicar até que ponto uma solução pode ser melhorada se uma restrição ou restrições forem relaxadas, de modo que o proprietário possa decidir se a restrição é necessária ou quanto vale a pena gastar para abolir a restrição.
        \end{quote}
        
        Com isso, podemos entender as restrições como limites impostos pelo sistema à alocação dos recursos. Baseado nelas, os objetivos serão alcançados respeitando ao máximos esses limites, ou então levarão em conta a possibilidade de se flexibilizar alguma delas de acordo com um peso respectivo.
    
    \subsection{\textit{Timetable} (grade horária)}
    
        Wren \cite{WREN1996} também cita que a grade horária em si apenas se refere a quando que determinado evento ocorrerá. Entretanto, também alerta quanto ao frequente uso do termo \textit{timetabling} na literatura como um sinônimo de \textit{scheduling}.
    
    \subsection{\textit{Class timetable}}
    
        Aqui especificamos um pouco mais quais são os recursos que deverão ser alocados, assim definindo que o contexto escolar como foco. Desse foco surgem algumas subáreas similares entre si, também categorizadas por Wren \cite{WREN1996}:
        
        \begin{itemize}
            \item A \textbf{grade horária de salas} durante o ensino de crianças, tende a muitas vezes ser um ou poucos professores que alternam entre si em uma mesma sala, assim não havendo maiores problemas na alocação de ambos os recursos.
            \item A \textbf{grade horária de provas} levarão em conta a atribuição de grupos de alunos que a realizarão, bem como possíveis salas especiais necessárias. Algumas restrições novas são consideradas nesse contexto, onde pode-se levar em conta a quantidade de provas que determinado aluno tem agendado para aquele dia ou semana, ou em períodos consecutivos.
            \item O foco deste trabalho é na \textbf{university class timetable} onde cada professor tem uma listagem de disciplinas que pode ministrar, eles também têm preferências de horários e cada aluno tem um conjunto de classes que pode participar, essas regras e preferências variam de instituição para instituição, podendo ter diferentes pesos a serem considerados para se alcançar o objetivo.
        \end{itemize}

\section{Interação Humano-Máquina}

    O maior enfoque na pesquisa literária realizada para este trabalho foi na área da Interação Humano-Máquina, onde alguns conceitos-chave podem ser melhor esclarecidos.
    
    \begin{itemize}
        \item \textbf{Interface}: Segundo Paula \cite{PAULA2011} "interface de um material é a superfície que possibilita o contato entre usuário e conteúdo, ou entre usuário e usuário (no caso de estarem conectados em rede) e permite o acesso às funções do sistema."
        \item \textbf{Design de Interação (\textit{Interaction Design - IxD})}: Andre \cite{ANDRE2018} descreve o Design de interação como sendo "o princípio de se desenvolver produtos digitais com uma abordagem centrada no usuário e envolve o usuário em todo o fluxo de trabalho."
        \item \textbf{Análise Visual (\textit{Visual Analytics})}: Para Thomas, \cite{THOMAS2009} "a análise visual é o processo iterativo que envolve a coleta de informações, o pré-processamento de dados, representação do conhecimento, interação e tomada de decisão".
    \end{itemize}
    