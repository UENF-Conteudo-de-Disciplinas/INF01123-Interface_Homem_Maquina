\chapter{Discussão}

Considerando o objetivo do trabalho como sendo a análise do cenário geral em que se encontra a relação do Timetabling com a Interação Humano-Computador, vemos cada um dos três artigos enfocados como relevantes nessa análise.

O artigo sobre o software udpSkeduler desenvolvido por \cite{MIRANDA2012505} traz à minha pesquisa a segurança na comprovação das benesses de se utilizar um sistema de suporte à decisão. Ele informa também sobre os resultados ótimos obtidos através do uso da programação inteira. Outro ponto deveras característico percebido durante seu desenvolvimento é a utilização de uma aplicação elaborada em diferentes módulos e etapas. Tendo elas tamanha relevância para tratar objetivamente sobre diferentes assuntos.

Enquanto isso, o artigo sobre dados de pré-processamento \cite{THOMAS2009} representa a complexidade intrínseca a um problema multidimensional, sendo então difícil conceber visual e mentalmente de que forma os dados relacionados ao problema se estruturam. Dessa forma, a apresentação de diversas formas de visualização deste artigo podem ter forte influência no desenvolvimento de uma aplicação que seja interativa e engaje o usuário. Também contribui ao ilustrar novas funcionalidades que podem auxiliar o usuário a ter insights para a elaboração da grade.

Por fim, o artigo sobre o \textit{Desing} de Interação \cite{ANDRE2018} descreve o fluxo necessário para criação de interfaces de focadas ao usuário, desde o modelo usado até o resultado da satisfação dos possíveis usuários. Também informando outras metodologias possíveis, mesmo que não escolhidas. O artigo ilustra diferentes modelos visuais para a representação das tabelas utilizadas no problema, ressaltando o impacto que a alteração de cores influencia na intuitividade do usuário.
