% --- Informações de dados para CAPA e FOLHA DE ROSTO --- %

\autor{João Vítor Fernandes Dias}
\titulo{Desenvolvimento centrado ao usuário como forma de viabilizar a resolução do problema de grade horária}
\local{Campos dos Goytacazes, RJ}
\data{\today}

\preambulo{Revisão literária desenvolvida para a disciplina de Interface Homem-Máquina da Universidade Estadual do Norte Fluminense Darcy Ribeiro.}
% \instituicao{Universidade Estadual do Norte Fluminense Darcy Ribeiro \par Ciência da Computação \par Metodologia de Pesquisa}
\orientador{Luis Antonio Rivera Escriba}
\tipotrabalho{Revisão Literária}
%\coorientador{Equipe \abnTeX}
% O preambulo deve conter o tipo do trabalho, o objetivo, o nome da instituição e a área de concentração
% ---


% ---
% Configurações de aparência do PDF final
% alterando o aspecto da cor azul
\definecolor{blue}{RGB}{41,5,195}

% informações do PDF
\makeatletter
\hypersetup{
%pagebackref=true,
    pdftitle={\@title}, 
    pdfauthor={\@author},
    pdfsubject={\imprimirpreambulo},
    pdfcreator={LaTeX with abnTeX2},
    pdfkeywords={abnt}{latex}{abntex}{abntex2}{trabalho acadêmico}, 
    colorlinks=true,       		% false: boxed links; true: colored links
    linkcolor=blue,          	% color of internal links
    citecolor=blue,        		% color of links to bibliography
    filecolor=magenta,      		% color of file links
    urlcolor=blue,
    bookmarksdepth=4
}
\makeatother