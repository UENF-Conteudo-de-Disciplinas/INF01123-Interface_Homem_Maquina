\chapter{Contextualização} % Apresentação das características dos artigos de referência narrativa do contexto do trabalho

Para este trabalho, foram três os artigos lidos:

\begin{itemize}
    \item \textit{udpSkeduler: A Web architecture based decision support system for course and classroom scheduling} \cite{MIRANDA2012505}
    \item \textit{Visualization Techniques on the Examination Timetabling Pre-processing Data} \cite{THOMAS2009}
    \item \textit{Interaction Design to Enhance UX of University Timetable Plotting System on Mobile Version} \cite{ANDRE2018}
\end{itemize}

\section{\textit{udpSkeduler DSS}}

Neste artigo os autores trazem comprovações através de dados experimentais quanto aos benefícios que um Sistema de Suporte à Decisão pode trazer ao Timetabling Problem. Além disso, é desenvolvido um sistema para geração de tabelas ótimas utilizando o modelo de programação inteira, permitindo interação direta dos professores com o sistema para que adicionem suas preferências de horários de aula. Os autores também explicam a importância dos Sistemas de Suporte à Decisão, utilizando da revisão bibliográfica.

O problema encontrado pelos autores era que, na universidade em questão, o processo era gerido por 3 gestores que geravam as grades de horário e um coordenador que trabalhava exclusivamente na designação de salas, sendo então um trabalho manual que usava o cronograma do semestre anterior como um formato base. Com isso, erros eram frequentes, assim causando inúmeros conflitos colocando muitas aulas em um mesmo horário, e sobrecarregando a disponibilidade de salas. Causando então o aluguel de salas externas o que gerava problemas administrativos.

Sua motivação era então automatizar parte desse processo, para assim evitar os conflitos recorrentes e liberar estes funcionários para realizarem outras tarefas. Consequentemente atingindo também maior satisfação quanto à grade horária.

Os objetivos a serem alcançados pelo sistema eram:

\begin{enumerate}
    \item Minimizar o uso de auditórios
    \item Minimizar a alocação de aulas laboratoriais fora do horário preferido
    \item Minimizar conflitos de alunos e matérias que precisam
    \item Minimizar o número de salas externas
    \item Maximizar o uso das preferências de tempo dos professores
\end{enumerate}

\section{\textit{Timetabling pre-processing data}}

O artigo em questão provê visualizações interativas úteis ao problema de de organização de horários de provas. Porém, apenas tratando dos "dados crus" (\textit{raw data}), visto que esses dados são os existentes prévios ao organizador que gera novas tabelas.

O artigo foca em usar conceitos de Interface Homem Computador (IHC) para realizar a ponte entre o usuário e o sistema, considerando que o usuário precisará explorar os dados e aprender a ter \textit{insights} que é o foco da visualização, sendo essa sua parte fundamental. O artigo trabalha bastante com o conceito de \textit{Visual Analytics} que é descrito como "a ciência do raciocínio analítico, facilitado por interfaces visuais interativas".

Ocorre a busca por viabilizar aos designers e tomadores de decisão da tabela de horários a análise dos dados que são relevantes, mesmo sendo multidimensionais e com muitas restrições, para que por fim possam tomar decisões efetivas. Com a adaptação da análise visual e com classificação de técnicas de interação, é esperado que a ciência de interação expanda com benefícios no núcleo dos designers de timetable. Seu objetivo é expandir o papel em que a interação engaja na \textbf{Infovis}, onde facilita na análise dos problemas de timetable. Também é visada a pesquisa de formas com que a interação é usada em \textbf{Infovis} e os benefícios gerados aos usuários dos sistemas de timetable.

\section{\textit{Interaction Design} (\textit{IxD})}

Este artigo conseguiu confirmar, que a abordagem prática guiada pela metodologia de desenvolvimento chamada Design de Interação tem sucesso em guiar o desenvolvedor ao design de melhores \textit{UX} e \textit{UI}

O artigo também cita conceitos como a avaliação de usabilidade e também vários modelos de fluxo de trabalho como Cascata, \textit{UX} Ágil, Esqueumorfismo, \textit{Five Design Sheet} e por fim aborda o que escolheu para seu desenvolvimento que é o Design de Interação (\textit{Interaction Design - IxD}).

O artigo buscou introduzir como a implementação do Design de Interação pode resolver o problema de UI/UX em dispositivos mobile, usando uma abordagem centrada no usuário para entregar engajamento. Aplicando isto no sistema de plotagem no quadro de horários da universidades e fazendo a Análise de usabilidade.
